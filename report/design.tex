\section{Design}
\subsection{Precursory Refactoring}

In order to provide a more clearer overview of the potential approaches to distributing the game, the provided single player code was refactored. This resulted in a clearer separation of the game logic into a clear structure. This approach also permitted the introduction of a test-suite that serve to guarantee the functionality as the application evolved into its distributed form.

This evolution allowed the creation of three key models: 

\begin{description}
	\item [Yahtzee]  Represents the entry point to a game of Yahtzee and the iteration aspect through the rounds until the completion of the game.
	\item [Round] Defines the logic behind each round; initial dice rolling, dice re-rolling logic and interfaces score board interaction.
	\item [Scoreboard] Represents a facade for the compound logic behind how scoring options are handled, aggregated and scored.
\end{description}

\subsection{Overview of Client-Server Architecture}

As with all client-server style applications, the server must first be started so that it can accept incoming connections from clients. Table \ref{tab:connectionProtocol} describes the connection protocol.

The life-cycle of a server instance and its players is defined as a Lobby. 

\begin{table}[H]
	\centering
	\begin{tabular}{ | l | l | }
		\hline
		(n) Client & (1) Server \\
		\hline
		\hline
		 & Server started \\
		 \hline
		 & While n clients are not connected and ready \\
		 \hline
		 Client started  & \\
		 \hline
		Client opens connection & \\
		 \hline
		 & Server accepts connection \\
		 \hline
		 & Server creates player from connection  \\
		 \hline
		Client chooses alias & \\
		 \hline
		& Server creates game instance \\
		\hline
		& Server notifies ready state  \\
		 \hline
		& End While \\
		\hline
	\end{tabular}
	\caption{Client connection process}
	\label{tab:connectionProtocol}
\end{table}

Every new connection results in a player session consisting of an input and output stream. A session is defined as a live connection. Each session has an associated session state: a game of Yahtzee. Each of these occupies a separate thread.

This effectively means that a player game session can survive independently of a connection being disconnected, allowing for the resumption of a game say if a client disconnection occurred through the simple implementation of a session identifier. %PoEAA 456

On the acceptance of each connection and successive creation each player by the lobby life-cycle, a series of shared dependencies are injected: the event bus and turn mediator. Within the player session, the connection streams are passed as abstract input and output streams to the Yahtzee game and beyond.

This game of Yahtzee relies on the injection of a Turn Mediator (Table \ref{tab:turnBasedMechanism}) implementation, responsible for arbitrating the order in which each game is played. The arbitration occurs during the iteration through the rounds, executing the round as appropriate.

The client application is a thin client, with little more than the definition of a connection protocol (Table \ref{tab:connectionProtocol}) , disconnection protocol (Table \ref{tab:disconnectionProtocol}) and input trigger protocol (Table: \ref{tab:inputProtocol}), documented through a shared library component model \it{ConnectionProtocol}.

\begin{table}[H]
	\centering
	\begin{tabular}{ | l | l | }
		\hline
		(n) Game Instance & (1) Turn Mediator \\
		\hline
		\hline
		 &  Create mediator \\
		\hline
		Register with mediation &  \\
		\hline
		& Determine first turn \\
		\hline 
		IF is my turn & \\
		\hline
		Request lock on turn & \\
		\hline
		Play round & \\
		\hline
		Release turn &  \\
		\hline
		ELSE wait until woken & \\ 
		\hline
		& Give next player turn using circular turning  \\
		\hline
	\end{tabular}
	\caption{Turn-based mechanism}
	\label{tab:turnBasedMechanism}
\end{table}

Communication between sessions is performed over the injected event bus, allowing players and the lobby to coordinate in a decoupled manner over actions. The event bus is an extension of the listener design pattern, allowing a publish-subscribe approach to communication between multiple threads. 

This intra-player, intra-thread model coordination is defined by sharing an Application Programming Interface package. 

\begin{table}[H]
	\centering
	\begin{tabular}{ | l | l | }
		\hline
		(n) Client & (1) Server  \\
		\hline
		\hline
		&  'INPUT' String sent alongside prompt  \\
		\hline
		Recognition of 'INPUT' String as trigger to require input & \\
		\hline
		Opens a local input reader &  \\
		\hline
		Reads input &  \\
		\hline
		Closes input & \\
		\hline
		Sends input & \\
		\hline
		 &  Handles input  according to context \\
		\hline
	\end{tabular}
	\caption{Input protocol}
	\label{tab:inputProtocol}
\end{table}

\begin{table}[H]
	\centering
	\begin{tabular}{ | l | l | }
				\hline
				(n) Client & (1) Server  \\
				\hline
				&  'TERMINATE CONNECTION' String sent \\
				\hline
				Recognition of 'TERMINATE CONNECTION' & \\ 
				String as trigger to close connection & \\
				\hline 
				Listen loop broken & \\
				\hline
				Connection closed & \\
				\hline
	\end{tabular}
	\caption{Disconnection Protocol}
	\label{tab:disconnectionProtocol}
\end{table}